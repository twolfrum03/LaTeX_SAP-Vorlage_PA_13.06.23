\chapter{Einleitung}

\section{Unternehmensprofil und Anwendungsbezug}

SAP ist ein börsennotierter Softwarekonzern mit Sitz in Walldorf. Das Unternehmen wurde 1972 von 5 IBM-Mitarbeitern, darunter Hasso Plattner und Dietmar Hopp gegründet. Das Hauptgeschäft ist die Entwicklung von Unternehmenssoftware zur Abwicklung von Geschäftsprozessen, unter anderem in den Berichen Buchführung, Controlling, Vertrieb, Einkauf, Produktion, Lagerhaltung, Transport und Personalwesen. Für das Unternehmen arbeiten heute 105.000 Mitarbeiter an Standorten in 157 Ländern und erwirtschaften einen Umsatz von ca. 29,5 Mrd. \euro{}. Erfolgreich wurde das Unternehmen mit seinem Standardsoftwarepaket SAP R/2 für Gro{\ss}rechnersysteme und später mit SAP R/3 für Client-Server-Systeme. Die Vorstellung der Mittelstandslösung SAP ByDesign im Jahr 2007 als Cloud-Produkt läutete die bis heute andauernde Transformation der gesamten Produktpallette in Richtung Cloud/ SaaS ein, die 2015 mit der Einführung von S/4 HANA als Hauptprodukt noch einmal verstärkt wurde.

Die Abteilung AIS HCM ist Teil des Product Engineering Unternehmensbereichs und zuständig für den 2nd-Level-Support und Eigenentwicklungen für die SAP on-Premise Personallösung HCM. Die Kunden der Abteilung sind Unternehmen die HCM verwenden und zusätzlich Wartungsverträge mit der SAP abgeschlossen haben. Zudem stellt die Abteilung mehrere SAP Fiori Apps als Self-Service für Mitarbeiter z.B. um Urlaub zu beantragen und Manager z.B. um Urlaubsanträge zu bearbeiten, bereit. Diese Apps sind für das Produkt HCM aufgrund des hohen Nutzungsvolumens von gro{\ss}er betriebswirtschaftlicher Bedeutung.

\section{Problemstellung}

Bestehendes Problem, Auslöser (Umstieg auf Business Objects? -> Was genau heißt das? Von was aus wird umgestiegen? Wieso wird umgestiegen?) erklären (Umstieg zu Fiori Elements, dadurch keine Logik mehr im Frontend, dadurch ist stateless angelegte REST-API ein problem für sequentielle Prozesse, die asynchrone Kommunikation verwenden)
-> Somit müssen Möglichkeiten, wie man diese Prozesse trotzdem umsetzen kann, ausloten



\section{Motivation}

Wieso ist das Thema bzw. die Arbeit wichtig?

\section{Abgrenzung}

Hier die Arbeit klar abgrenzen, also was genau behandelt wird und eben was auch nicht betrachtet wird (z.b.) das keine Implementierung der vorgestellten Ansätze erfolgen soll, (wenn dann nur kleiner Prototyp)
Ggf. auch kurz darauf eingehen welche Ansätze auch nicht beleutet werden

\section{Methodisches Vorgehen}

Wie werden die Ansätze betrachtet? Wie kommt die Handlungsempfehlung zustande?