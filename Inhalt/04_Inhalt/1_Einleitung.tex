\chapter{Einleitung}

\section{Problemstellung}

Das Bewusstsein für Umweltprobleme hat sich weltweit erhöht und immer mehr Menschen fordern Unternehmen auf, Verantwortung für ihre Auswirkungen auf die Umwelt zu übernehmen.
Gleichzeitig setzen Regierungen und internationale Organisationen verstärkt auf Nachhaltigkeit und haben Nachhaltigkeitsziele formuliert.
Auch Investoren und Finanzinstitute haben begonnen, Nachhaltigkeitsaspekte in ihre Entscheidungen einzubeziehen und fordern von Unternehmen eine bessere Berücksichtigung von Umweltfaktoren.
All diese Entwicklungen haben dazu geführt, dass Unternehmen, zunehmend unter Druck geraten und mit Image- und Reputationsproblemen sowie regulatorischen Risiken konfrontiert sind. 

\section{Motivation}

Applikationen zur Optimierung von Geschäftsprozessen werden seit je her benutzt, um Prozesse zu verbessern und an neue Anforderungen anzupassen.
Die Integration von relevanten Werten und Indikatoren zum Themengebiet Nachhaltigkeit in einem solchen Tool könnte Unternehmen dabei helfen
ihre Prozesse ökologisch nachhaltiger zu gestalten und damit den steigenden Anforderungen an nachhaltige Geschäftspraktiken gerecht zu werden.

\section{Methodik und Vorgehen}

Die Methodik orientiert sich am Ansatz Design Science Research nach Hevner.
Im ersten Teil der Arbeit werden die Grundlagen zum Thema Nachhaltigkeit für Unternehmen erläutert und relevante Literatur herangezogen.
Anschließend werden die Nachhaltigkeitsindikatoren ermittelt, die für die Bewertung der Nachhaltigkeit von Geschäftsprozessen relevant sind.
Es wird untersucht, welche Datenquellen für die Erhebung, Generierung oder Beschaffung dieser Daten genutzt werden können.
Basierend auf diesen Erkenntnissen wird ein Konzept, Modell oder Prototyp erarbeitet,
der die Integration dieser Indikatoren in Applikationen zur Geschäftsprozessoptimierung am Beispiel der SAP Signavio Process Insights darstellt.