\chapter{Einleitung}

\section{Unternehmensprofil und Anwendungsbezug}

SAP SE ist ein börsennotierter Softwarekonzern mit Sitz in Walldorf. Das Hauptgeschäft des 1972 gegründeten Unternehmens ist die Entwicklung von Unternehmenssoftware zur Abwicklung von Geschäftsprozessen. Heute erwirtschaften 105.000 Mitarbeiter in 157 Ländern einen Umsatz von ca. 30 Mrd. \euro{}. Erfolgreich wurde das Unternehmen mit dem Verkauf von ERP Standardsoftware. In den letzten Jahren stand die Transformation des gesamten Produkt-Portfolios in Richtung Cloud-Services als Abo-Modell im Fokus der Unternehmensstrategie. \footcite[Vgl.][]{sap_geschichte_2023}

Die Abteilung AIS HCM ist Teil des Unternehmensbereichs Product Engineering und zuständig für den Development-Support und Neuentwicklungen der SAP Personallösung HCM. Diese deckt Prozesse rund um das Personalwesen ab. Zudem stellt die Abteilung mehrere SAP Fiori Apps als Self-Service für Mitarbeiter bereit. Durch das hohe Nutzungsvolumen dieser Apps und der somit gro{\ss}en betriebswirtschaftlichen Relevanz, sind diese und auch der Untersuchungsgegenstand dieser Arbeit für das Produkt HCM von gro{\ss}er Bedeutung.

% - SAP-Personallösung HCM als Teil des ERP-Systems ECC (alt) und S/4 HANA (neu)
% - Abdeckung Software rund um Personalwesen , -planung (Abwesenheiten, Zeiten)
% - Keine Unternehmenshierarchie
% - Abteilung: Development-Suppport, Analyse Systemlandschaft, Dimensionierung
% - ECC, S4 on-prem Editionen -> Wartung für Kunden mit Wartungsverträgen, Neuentwicklung für HCM
% - Arbeit im Kontext der SAP HCM Anwendung, die Teil von SAP ECC, S/4 ist

\section{Motivation und Problemstellung}

Die von der Abteilung betriebenen Fiori Apps, die schon im Zusammenhang der Einleitung angesprochen wurden, sind auf Basis des Frameworks SAP UI5 Freestyle für ein älteres Produkt - SAP ECC - entwickelt worden. Durch die strategische Entscheidung HCM im neuen S/4 HANA System (''S/4'' abgekürzt) durch die neue cloudbasierte Personallösung SuccessFactors abzulösen war dieser Umstand ursprünglich kein Problem. Diese Entscheidung wurde aufgrund fehlender Funktionalitäten in SuccessFactors und hoher verlässlicher Einnahmen durch Wartungsverträge für HCM revidiert und HCM ist Bestandteil von S/4. Somit finden die S/4-Design-Guidelines darauf Anwendung, die \zB Oberflächen-Design oder zu verwendende Technologien, festlegen. Fiori Apps müssen dadurch die Technologie Fiori Elements verwenden. Aus Praktikabilitätsgründen dürfen existierende Apps auf Basis der älteren Technologie in S/4 weiterbetrieben werden und nur neu entwickelte Apps müssen Fiori Elements verwenden.

Diese Situation sorgt für ein Problem in Geschäftsprozessen, die über solche Apps abgebildet werden sollen. Das Framework Fiori Elements generiert das gesamte Front-End der Anwendung selbstständig. Dadurch ist Das erleichtert auf der einen Seite die Entwicklung der Apps, auf der anderen Seite kann dadurch keine eigene Programmlogik mehr im Front-End eingebaut werden. Zudem bietet das Programmiermodell RAP, das in Fiori Elements für eine konsistente Durchführung der Datenbankoperationen zuständig ist, nur einen transaktionalen Kontext für das Ausführen von eigener Logik. Jedoch ist es in bestimmten Geschäftsprozessen nötig, nachgelagert in einem weiteren transaktionalen Kontext noch Programmcode ausführen zu können. Da RAP das modellseitig nicht zulässt, soll in der vorliegenden Arbeit soll nun untersucht werden, wie sich solche asynchronen Prozesse, trotz den eben dargelegten Einschränkungen im S/4 Umfeld mit den neueren Technologien umsetzen lassen.

% - SAP UI5 ist open-source UI-Bibliothek -> Fiori als darauf aufbauendes Framework zur Entwicklung von Apps
% - Fiori gibt es als Freestyle und in S/4 als Elements Variante
% - Elements sehr starr, nur ein transaktionaler Kontext, daneben keine weitere asynchrone Code-Ausführung ->Problem
% - RAP stellt im Kontext der Fiori App sicher, dass Datenbankoperationen innerhalb einem LuW konsistent durchgeführt wird
% - Brauchen nebem "Haupt LUW" noch anderen LUW der zb Emails versendet oder asynchron noch Code ausführt -> geht nicht in RAP/ Fiori Elements ->Aufruf von "2. Programm"
% - Transaktionaler Kontext: LuW
% - keine Treiber der strategischen Entscheidung nennen
% - "HCM ist in S/4, deswegen Frameworks, ..., deswegen Probleme

\section{Aufbau und Ziel der Arbeit}

Als erstes wird im einleitenden Kapitel die SAP und die Abteilung AIS HCM, in der die Praxisphase absolviert wurde, kurz vorgestellt und somit der Anwendungsbezug der Arbeit hergestellt. Danach wird mit der Motivation und Problemstellung der Untersuchungsgegenstand und die Bedeutung der Arbeit für die Abteilung und die Kunden erläutert. Danach soll die Arbeit klar von verwandten Themen abgegrenzt werden, um einen klaren Rahmen für die Untersuchung zu schaffen. Im methodischen Vorgehen werden dann abschlie{\ss}end für die Einleitung noch auf die wissenschaftlichen Methoden, die verwendet wurden, um die Untersuchungsergebnisse zu erhalten, eingegangen. Der Hauptteil der Arbeit besteht aus zwei Teilen: Im ersten Teil werden die theoretischen Grundlagen der Arbeit gelegt. Hier werden die Designprinzipien einer RESTful API, wie diese im RESTful Application Programming Model der SAP eingesetzt werden und die Technologie Fiori Elements näher beleuchtet. Der praktische Hauptteil stellt drei Ansätze vor, wie das in der Problemstellung thematisierte Problem gelöst werden kann. Hierfür werden die Technologien Business Workflows, Business Events und das Background Processing Framework vorgestellt und im Bezug auf Stärken und Schwächen sowie Effizienz und Robustheit verglichen. Diese Ergebnisse werden dann in einer Entscheidungsmatrix dargestellt. Das Ziel soll es sein, dass diese Entscheidungsmatrix klare Tendenzen gibt, welche der untersuchten Technologien sich in einem konkreten Anwendungsfall für das Abbilden von asynchronen Prozessen im RESTful API Umfeld anbietet. Im Schlussteil werden die Ergebnisse der Arbeit nochmals zusammengefasst, eine konkrete Handlungsempfehlung für die Lösung dieses Problems gegeben und die Ergebnisse abschlie{\ss}end kritisch reflektiert und ein Ausblick auf zukünftige Entwicklungen gegeben.

\section{Abgrenzung}

Der Zweck der vorliegenden Arbeit ist es, die drei vorgestellten Technologien vergleichend zu bewerten und je nach Anwendungsfall eine Handlungsempfehlung im Bezug auf eine sich anbietende Technologie zu geben. Über diese drei Technologien hinaus werden keine anderen Möglichkeiten asynchrone Prozesse abzubilden, wie \zB im Cloud Application Programming Model (CAP) behandelt. Au{\ss}erdem findet aufgrund des beschränkten Umfangs der Arbeit lediglich ein Vergleich der Technologien statt und keine direkte Implementierung dieser in einem konkreten Anwendungsfall. Hierfür sei auf die offizielle Dokumentation der SAP mit Showcases für die respektiven Technologien verwiesen.

\section{Methodisches Vorgehen}

Nachdem die drei Ansätze vorgestellt wurden, sollen diese im Bezug auf mehrere Kriterien betrachtet und anhand dieser miteinander verglichen werden. Diese Kriterien werden bei den einzelnen Ansätzen durch Experteninterviews bewertet und dann die betrachteten Ansätze anhand dieser Kriterien gegenübergestellt. So kommt dann die Entscheidungsmatrix zu Stande, welche Technologie sich bei welchen Anforderungen und Rahmenbedingungen anbietet.

Die Methode des Experteninterviews bietet sich an, weil sich das Thema der Arbeit auf spezifische SAP-Technologien bezieht. Diese Technologien können von Fachleuten auf diesem Gebiet gut umrissen und im Bezug auf die Vergleichskriterien eingeordnet werden.

Experteninterviews sind eine qualitative Forschungsmethode, durch die eine umfassende Einsicht und persönliche Perspektive von mit dem Thema vertrauten Personen erfasst. Die direkte Kommunikation erleichtert das Sammeln von Informationen, die in der Literatur zu unspezifisch oder nicht ausreichend vorkommen. Dies ist besonders bei praxisnahen Themen wichtig, da hier Best Practices und bekannte Herausforderungen ermittelt und daraus auf spezielle Anwendungsfälle passende Ansätze erarbeitet werden können.

Zur Auswahl stehen ein strukturiertes und unstrukturiertes Experteninterview. Für die hier durchgeführten Interviews wird eine Mischform aus beiden gewählt. In der Vorbereitung des Interviews wird strukturiert vorgegangen und es werden klare Fragen formuliert. Die Durchführung des Interviews ist eher unstrukturiert, damit die Reihenfolge der Fragen geändert werden kann und bei Bedarf auch noch flexibel Rückfragen gestellt werden können.