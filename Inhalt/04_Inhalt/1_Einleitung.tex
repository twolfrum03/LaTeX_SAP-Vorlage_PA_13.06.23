\chapter{Einleitung}

\section{Unternehmensprofil und Anwendungsbezug}

SAP SE ist ein börsennotierter Softwarekonzern mit Sitz in Walldorf. Das Hauptgeschäft des 1972 gegründeten Unternehmens ist die Entwicklung von Unternehmenssoftware zur Abwicklung von Geschäftsprozessen. Heute erwirtschaften 105.000 Mitarbeiter in 157 Ländern einen Umsatz von ca. 30 Mrd. \euro{}. Erfolgreich wurde das Unternehmen mit dem Verkauf von ERP Standardsoftware. In den letzten Jahren stand die Transformation des gesamten Produkt-Portfolios in Richtung Cloud-Services als Abo-Modell im Fokus der Unternehmensstrategie. \footcite[Vgl.][]{sap_geschichte_2023}

Die Abteilung AIS HCM ist Teil des Unternehmensbereichs Product Engineering und zuständig für 2nd-Level-Support und Eigenentwicklungen der SAP Personallösung HCM. Zudem stellt die Abteilung mehrere SAP Fiori Apps als Self-Service für Mitarbeiter bereit. Durch das hohe Nutzungsvolumen dieser Apps und der somit gro{\ss}en betriebswirtschaftlichen Relevanz, sind diese und auch der Untersuchungsgegenstand dieser Arbeit für das Produkt HCM von gro{\ss}er Bedeutung.

\section{Motivation und Problemstellung}

Im folgenden Kapitel soll dargestellt werden, welche Probleme sich durch gewisse technische Veränderungen der SAP Produkte ergeben und sich somit für eine wissenschaftliche Untersuchung im Rahmen dieser Arbeit anbieten.

Die von der Abteilung betriebenen Fiori Apps, die schon im Zusammenhang der Einleitung angesprochen wurden, sind auf Basis des Frameworks SAP UI5 Freestyle für ein älteres Produkt - SAP ERP - entwickelt worden. Durch die strategische Entscheidung HCM im neuen S/4 HANA System (''S/4'' abgekürzt) durch die neue cloudbasierte Personallösung SuccessFactors abzulösen war dieser Umstand ursprünglich kein Problem. Diese Entscheidung wurde aufgrund fehlender Funktionalitäten in SuccessFactors und hoher verlässlicher Einnahmen durch Wartungsverträge für HCM revidiert und HCM ist Bestandteil von S/4. Somit finden die S/4-Design-Guidelines darauf Anwendung, die \zB Oberflächen-Design oder zu verwendende Technologien, festlegen. Fiori Apps müssen dadurch die Technologie Fiori Elements verwenden. Aus Praktikabilitätsgründen dürfen existierende Apps auf Basis der älteren Technologie in S/4 weiterbetrieben werden und nur neu entwickelte Apps müssen Fiori Elements verwenden.

Diese Situation sorgt für ein Problem in Geschäftsprozessen, die über solche Apps abgebildet werden sollen. Das Framework Fiori Elements generiert das gesamte Front-End der Anwendung selbstständig. Das erleichtert auf der einen Seite die Entwicklung der Apps, auf der anderen Seite kann dadurch keine eigene Programmlogik mehr im Front-End eingebaut werden. Zudem wird die Kommunikation mit dem Back-End über eine RESTful API, die zustandslos angelegt ist, abgewickelt. Auch wenn eine RESTful-API viele Vorteile mit sich bringt, sind Anwendungen, deren Prozesse asynchrone Kommunikation benötigen nur noch schwer abbildbar.

In der vorliegenden Arbeit soll nun untersucht werden, wie sich solche asynchronen Prozesse, trotz den eben dargelegten Einschränkungen trotzdem im neuen S/4 HANA Umfeld mit den neueren Technologien umsetzen lassen.

\section{Aufbau und Ziel der Arbeit}

Im Folgenden wird der Aufbau und das Ziel der Arbeit thematisiert.

Als erstes wird im einleitenden Kapitel die SAP und die Abteilung AIS HCM, in der die Praxisphase absolviert wurde, kurz vorgestellt und somit der Anwendungsbezug der Arbeit hergestellt. Danach wird mit der Motivation und Problemstellung der Untersuchungsgegenstand und die Bedeutung der Arbeit für die Abteilung und die Kunden erläutert. Danach soll die Arbeit klar von verwandten Themen abgegrenzt werden, um einen klaren Rahmen für die Untersuchung zu schaffen. Im methodischen Vorgehen werden dann abschlie{\ss}end für die Einleitung noch auf die wissenschaftlichen Methoden, die verwendet wurden, um die Untersuchungsergebnisse zu erhalten, eingegangen. Der Hauptteil der Arbeit besteht aus zwei Teilen: Im ersten Teil werden die theoretischen Grundlagen der Arbeit gelegt. Hier werden die Designprinzipien einer RESTful API, wie diese im RESTful Application Programming Model der SAP eingesetzt werden und die Technologie Fiori Elements näher beleuchtet. Der praktische Hauptteil stellt drei Ansätze vor, wie das in der Problemstellung thematisierte Problem gelöst werden kann. Hierfür werden die Technologien Business Workflows, Business Events und das Background Processing Framework vorgestellt und im Bezug auf Stärken und Schächen sowie Effizienz und Robustheit verglichen. Diese Ergebnisse werden dann in einer Entscheidungsmatrix dargestellt. Das Ziel soll es sein, dass diese Entscheidungsmatrix klare Tendenzen gibt, welche der untersuchten Technologien sich in einem konkreten Anwendungsfall für das Abbilden von asynchronen Prozessen im RESTful API Umfeld anbietet. Im Schlussteil werden die Ergebnisse der Arbeit nochmals zusammengefasst, eine konkrete Handlungsempfehlung für die Lösung dieses Problems gegeben und die Ergebnisse abschlie{\ss}end kritisch Reflektiert und ein Ausblick auf zukünftige Entwicklungen gegeben.

\section{Abgrenzung}

Der Zweck der vorliegenden Arbeit ist es, die drei vorgestellten Technologien vergleichend zu bewerten und je nach Anwendungsfall eine Handlungsempfehlung im Bezug auf eine sich anbietende Technologie zu geben. Über diese drei Technologien hinaus werden keine anderen Möglichkeiten asynchrone Prozesse abzubilden, wie \zB im Cloud Application Programming Model (CAP) behandelt. Au{\ss}erdem findet aufgrund des beschränkten Umfangs der Arbeit lediglich ein Vergleich der Technologien statt und keine direkte Implementierung dieser in einem konkreten Anwendungsfall. Hierfür sei auf die offizielle Dokumentation der SAP mit Showcases für die respektiven Technologien verwiesen.

\section{Methodisches Vorgehen}

Nachdem die drei Ansätze vorgestellt wurden, sollen diese im Bezug auf mehrere Kriterien betrachtet und anhand dieser miteinander verglichen werden. Diese Kriterien werden bei den einzelnen Ansätzen durch Experteninterviews bewertet und dann die betrachteten Ansätze anhand dieser Kriterien gegenübergestellt. So kommt dann die Entscheidungsmatrix zu Stande, welche Technologie sich bei welchen Anforderungen und Rahmenbedingungen anbietet.

