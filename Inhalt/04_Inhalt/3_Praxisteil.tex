\chapter{Praktischer Teil}

Im Folgenden werden die verschiedene praktische Lösungsansätze vorgestellt und anhand verschiedener Kriterien gegeneinander abgewogen, sodass am Ende eine Handlungsmatrix erstellt werden kann.

\section{Lösungsansätze}

Zunächst sollen drei verschiedene Technologien vorgestellt werden, mit denen sich asynchrone Prozesse mit sequentieller Kommunikation, trotz den Einschränkungen durch RAP und Fiori Elements, umsetzen lassen.

\subsection{Business Workflows}

Der erste mögliche Ansatz sind Business Workflows (BW abgekürzt).

\subsection{Business Events}

Eine Option wären die Verwendung von Business Events. Hier auch ggf. auf Probleme mit Event-Mesh (Cloud- bzw. BTP-Komponente) für onPremise-Systeme eingehen -> lokale Verarbeitung der Business Events?

\subsection{Background Processing Framework}

Andere Option wäre das Background Processing Framework über Background remote function calls.

\section{Entscheidungsmatrix}

Hier soll eine Entscheidungsmatrix entwickelt werden, welchen Lösungsansatz man in Abhängigkeit von mehreren Faktoren am besten verwenden soll (ersetzt auch weng mit die Zusammenfassung)

Vergleichskriterien:

- BTP Event Mesh Cloud nötig in Systemlandschaft, andere Lösungen laufen nur lokal -> mehr Kosten,  Komplexität in Systemlandschaft, Datenschutz
- Kosten
- Performance (wshl verlieren BusinessEvents, Kommunikation über Systemgrenzen hinaus)
- Experteninterview: BusinessEvents (Martin Müller (+pgpf), Marcel Herrmanns, Daniel Wachs) 