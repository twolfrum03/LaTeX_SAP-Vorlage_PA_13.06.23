\chapter{Einleitung}

\section{Motivation und Problemstellung}

Allein in den USA geben Unternehmen jährlich mehr als 200 Milliarden Dollar für Schulungen, Fortbildungen und Seminare aus \parencite[vgl.][]{LloydMewkirk.2011}.
Leider neigen Teilnehmer nach Abschluss jener Veranstaltungen dazu, gelerntes Wissen über die Zeit zu vergessen.
Ein Umstand, der weithin als Vergessenskurve bezeichnet wird und um 1885 vom deutschen Psychologen Hermann Ebbinghaus beschrieben wurde \parencite[vgl.][]{Ebbinghaus.1885}. 

Unternehmen investieren also viele Millionen Dollar in Lernveranstaltungen, nur damit vieles danach wieder in Vergessenheit gerät.
Ein dynamischer Learning-\acs{NFT} versucht dieses Problem nachhaltig, durch anschließende zeitlich versetzte Lerneinheiten zu minimieren und diesen Prozess analysier- und steuerbar zu machen. 
Des Weiteren gibt ein \acf{NFT} die Möglichkeiten verbesserter Fälschungssicherheit und bietet Nutzungsmöglichkeiten auch plattformübergreifend.

\section{Methodik und Vorgehen}

Im ersten Teil der Arbeit sollen die zugrunde liegenden Technologien erklärt und beschrieben werden.
Beginnend mit Grundlagen zur Distributed Leger Technologie und Blockchain soll die Funktionsweise von Smart Contracts erläutert werden.
Auf dieses Gerüst aufbauend werden dann \ac{NFT}s erklärt, sowie eine Abgrenzung zum gern Synonym verwendeten Thema Kryptowährungen vorgenommen werden.

Anschließend sollen Methoden und Ansätze des dynamischen Learning-\ac{NFT}s im Rahmen einer Fallstudie analysiert werden um Potenziale und Adaptionsmöglichkeiten für die Anwendung
auf der SAP Experience Garage Plattform herauszustellen und aufzuzeigen welche Verbesserungsmöglichkeiten die einzelnen Komponenten eines solchen
Learning-\ac{NFT} für die Experience Garage Plattform zukünftig haben könnten und ob sich eine Implementierung des Konzepts im Ganzen,
oder in Teilen, für die SAP Experience Garage lohnt.
Weiterhin könnten spezielle Anforderungen der Plattform eine Erweiterung beziehungsweise Modifikation des Konzepts erfordern beziehungsweise lohnenswert machen.

Abschließend soll nach einer kurzen Auseinandersetzung mit den Schattenseiten der Blockchain Technologie ein Fazit,
zum Adaptionspotenzial eines dynamischen Learning-\ac{NFT}s in die Experience Garage Plattform vorgenommen
und einen Ausblick auf mögliche Anwendung im unternehmensweiten Kontext gegeben werden.

\section{Zielsetzung}

Ziel ist es, die Ansätze und Methoden des Learning-\ac{NFT}s zu analysieren, um Potenziale und Probleme herauszustellen.
Des Weiteren sollen Möglichkeiten zur Adaption eines solchen Konzepts für die Experience Garage Plattform aufgezeigt und bewertet werden.